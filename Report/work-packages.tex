% Breakdown the work to subtasks to meet the project requirements.
	% Define and describe these tasks.
	
		\subsection{API}
			Jakob, Manuel

\begin{enumerate}
% @Manu: Hey Manu, i schrib do gad ine was i ändere, wenns glese hesch und kei Iwänd hesch, bitte gad wieder lösche ;)

% @Manu: Das item hani no hinzuegfüegt, will mer da bi de Milestones denn sicher bruchet
\item Define all public function signatures of our API and hand it to the other group members
\item Establish Peer-to-Peer connection with WiFi Direct
\item Build data structures for LC-Table and ACK-Table
% @Manu: Zwei neue items
\item Implement network initialization with two nodes
\item Write code for nodes to join an existing network

\item Implement functions to update tables
\item Build message and parse message
% @Manu: I würd do no kein gnaui Funktionenäme verwende, die müemer zerscht no definiere, also würdi do kei `Code' ine tue
% OLD \item Implement broadcast\_message
\item Implement message sending (broadcast)

% @Manu: send_message - Bruchet mer da würklich? Inwiefern bringt üs da überhaup öpiis, wenn mer wüsset das e Nochricht nurzu eim mues? Verschicke tümers doch eh allne, damits so schnell wie möglich achunt. Und wie chömmer da speichere ohni dass mer üsen ACK-Table kaputt machet, bruchet mer en Zweite ACK-Table extra für da? I wär defür, das mer da weglönd, aber me chönds gern bespreche, je nochdem sötter mer au no di anderne froge was meinet.
% OLD\item Implement send\_message

% OLD: \item Implement receive\_message (with message listener)
\item Implement message receiving (with message listener)
\item Build data structure for local buffer
% @Manu: Au no neu:
\item Implement buffer entry replacement strategy
\item Remove old nodes from network according to a timeout specified by the client
\item Correct reconnection
\item Request all buffered messages from a node
\item Unsubscribing nodes from the network
\end{enumerate}


		
		\subsection{Emergency App}
			Claude, Alessandro
			\begin{enumerate}
				\item Main Activity with "request help" button. Button is only clickable if personal information is entered and location services are turned on. On button click the user can select what kind of emergency case it is.
				\item Settings Activity which stores personal information such as Name, insurance numbers, allergies, etc.
				\item Notification Activity which shows a relayed emergency request on the users phone including walk directions to find the requester.
				\item A webserver which distributes the request to the specific emergency services in charge.
			\end{enumerate}
		
		\subsection{Chat App}
			Joel, Pascal
			\begin{enumerate}
				\item MainActivity: Clickable list of chats ordered by activity with names and unread message counter overflow menu with "Preferences", "Show Key", "Add Chat", "Go Offline"
				\item Storing chats, address book and own keys in files when service is shut down
				\item ChatActivity: Chat window, with message list left and right aligned, depending on sender, ordered descending in age
				\item Using ZXing library make two activities, one for displaying keys and one for scanning them
				\item Preferences, for sound and vibration and new key generation
				\item Service that handles message state, address book state, receiving messages including decryption, notification to be started on app start
				\item Generating public-private key pair with javax.crypto
				\item Activity for initial key generation and name entry
			\end{enumerate}
			
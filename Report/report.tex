% This is based on "sig-alternate.tex" V1.9 April 2009
% This file should be compiled with V2.4 of "sig-alternate.cls" April 2009
%
\documentclass{sig-alternate}

\usepackage[english]{babel}
\usepackage{graphicx}
\usepackage{tabularx}
\usepackage{subfigure}
\usepackage{enumitem}
\usepackage{url}
\usepackage{colortbl}
\usepackage{multirow}


\usepackage{color}
\definecolor{orange}{rgb}{1,0.5,0}
\definecolor{lightgray}{rgb}{.9,.9,.9}
\definecolor{java_keyword}{rgb}{0.37, 0.08, 0.25}
\definecolor{java_string}{rgb}{0.06, 0.10, 0.98}
\definecolor{java_comment}{rgb}{0.12, 0.38, 0.18}
\definecolor{java_doc}{rgb}{0.25,0.35,0.75}

% code listings

\usepackage{listings}
\lstloadlanguages{Java}
\lstset{
	language=Java,
	basicstyle=\scriptsize\ttfamily,
	backgroundcolor=\color{lightgray},
	keywordstyle=\color{java_keyword}\bfseries,
	stringstyle=\color{java_string},
	commentstyle=\color{java_comment},
	morecomment=[s][\color{java_doc}]{/**}{*/},
	tabsize=2,
	showtabs=false,
	extendedchars=true,
	showstringspaces=false,
	showspaces=false,
	breaklines=true,
	numbers=left,
	numberstyle=\tiny,
	numbersep=6pt,
	xleftmargin=3pt,
	xrightmargin=3pt,
	framexleftmargin=3pt,
	framexrightmargin=3pt,
	captionpos=b
}

% Disable single lines at the start of a paragraph (Schusterjungen)

\clubpenalty = 10000

% Disable single lines at the end of a paragraph (Hurenkinder)

\widowpenalty = 10000
\displaywidowpenalty = 10000
 
% allows for colored, easy-to-find todos

\newcommand{\todo}[1]{\textsf{\textbf{\textcolor{orange}{[[#1]]}}}}

% consistent references: use these instead of \label and \ref

\newcommand{\lsec}[1]{\label{sec:#1}}
\newcommand{\lssec}[1]{\label{ssec:#1}}
\newcommand{\lfig}[1]{\label{fig:#1}}
\newcommand{\ltab}[1]{\label{tab:#1}}
\newcommand{\rsec}[1]{Section~\ref{sec:#1}}
\newcommand{\rssec}[1]{Section~\ref{ssec:#1}}
\newcommand{\rfig}[1]{Figure~\ref{fig:#1}}
\newcommand{\rtab}[1]{Table~\ref{tab:#1}}
\newcommand{\rlst}[1]{Listing~\ref{#1}}

\title{WiFi Direct Message Flooding API \\
\normalsize{Distributed Systems -- Project Proposal}}
% \subtitle{subtitle}

% Use the \alignauthor commands to handle the names
% and affiliations for an 'aesthetic maximum' of six authors.

\numberofauthors{1} %  in this sample file, there are a *total*
% of EIGHT authors. SIX appear on the 'first-page' (for formatting
% reasons) and the remaining two appear in the \additionalauthors section.
%
\author{
% You can go ahead and credit any number of authors here,
% e.g. one 'row of three' or two rows (consisting of one row of three
% and a second row of one, two or three).
%
% The command \alignauthor (no curly braces needed) should
% precede each author name, affiliation/snail-mail address and
% e-mail address. Additionally, tag each line of
% affiliation/address with \affaddr, and tag the
% e-mail address with \email.
%
% 1st. author
%TODO: Everyone write up in this list:
\alignauthor \normalsize{Student One,  Student Two, Student Three, Student One,  Student Two, Jakob Meier}\\
	\affaddr{\normalsize{ETH ID-1 XX-XXX-XXX, ETH ID-2 XX-XXX-XXX, ETH ID-3 XX-XXX-XXX, ETH ID-1 XX-XXX-XXX, ETH ID-2 XX-XXX-XXX, 13-925-573}}\\
	\email{\normalsize{one@student.ethz.ch, two@student.ethz.ch, three@student.ethz.ch, one@student.ethz.ch, two@student.ethz.ch, jakmeier@student.ethz.ch}}
}

\begin{document}
	
	\maketitle
	
	\begin{abstract}
		Again, its this time of the year, when all the large festivals and parades are. Its this time, when all of us give in on one of these and go there, like every year. 
You go there with a few friends and soon enough you will loose one of them, because the crowd is just too big and too loud. Every one of us knows this situation, right? \\
It's this situation when you really need to use your mobile phone, but ending up be annoyed about the absence of ANY reception. \\
So that's the point where we started... \\

We want a messaging system to work, even if you can't reach youre friends over the internet. We also wanted not only to program a messaging app for that purpose, rather then go a step further and build an API to provide these functionalities to users with even different approches than a "normal" messaging app. \\

Our approach to build such a network of nodes in an usable range for mobile devices, is to use WiFi Direct. The API will not only forward the messages to a server, it broadcasts the messages to all nodes. This gives us the highest possibility to get a message to a node without any reception. \\

So far, so good. But it's not finished yet. It's a lot more stuff needed then just broadcast a message through a network of devices. The API should allow a dynamic network structure, which means at any time a node can leave or join the network, it also should allow a kind of buffering the messages to allow reaching nodes which are not reachable for the network at that moment. This leads to a lot of challenging problems with replacement orderings, timeouts and so forth... \\

---------------------------------------------------------------------------\\
TODO: Write here if i forgot something.... !! \\

----------------------------------------------------------------------------
	\end{abstract}
	\section{Introduction}
		Our message flooding API can be useful to many future projects that involve several Android devices which should be connected even without a working internet connection. For some applications, the API might simply provide an alternative communication channel that can be used when the device does not have a connection to the internet, but for other applications it can be the core of the communication between several devices. \\
A simple example application will be distributed along with the API as a demo. The demo is an SOS forwarding app that uses our API to propagate an emergency call between devices which are not connected with the internet, until it reaches a device with a working internet connection that can send the call to a webserver. \\
Of course the full power of the API will only be visible in more complex systems. In principle, the API will be powerful enough to support a document editor which is synchronized over many users, all without the need of a working internet connection. That could be interesting for a military office outside, but also for a working team that wants to keep working on the same files while traveling together in an airplane.\\
To demonstrate how the API is used for more complex applications, we will develop a messenger app. The app will support multiple secure chats that users can join. \\
As the name suggest, the API provides nothing but a message flooding interface, therefore most of the complexity will be in the client's code outside of the API, namely in the client's application. However, the API solves most of the problems of a distributed systems and hides them from the client. The features available in the API are:
\begin{itemize}
	\item {\bf Dynamic local network}: Devices can form a local network and new devices can enter it dynamically.
	\item {\bf Message flooding}: A device can easily send a message to all other devices in the local network.
	\item {\bf Message buffering}: A device which loses connection to the other devices will receive all sent messages when it connects to the local network again.
	\item {\bf Message reordering}: The ordering of messages sent by one device is preserved on the receiver side.
\end{itemize}
There are already applications and services which, to some degree, do the same. One such example is the FireChat\cite{FireChat} app. FireChat is a proprietary mobile app that builds on a wireless decentralized mesh network to enable smartphones to connect and send messages to each other. The main difference is that we develop an API instead of one single application. While our API can be used to implement something similar to FireChat (as we will show with our Chat App example) it is capable of handling many more and very different use cases like the aforementioned SOS app.

		
		% TODO: Describe similar projects like firechat and serval project
		
	\section{System Overview}
		% This is the core of the proposal.
	% It is where you spell out your technical plan and explain the project design.
	% Expected evaluation/demonstration issues would also be addressed in this section.
	% Use helpful figures such as~\rfig{example} and~\rfig{system-overview},
	% explain the figures in the text where you reference them. 
		\subsection{API}
		The API will offer the following functionalities to clients:
		\begin{itemize}
			\item {Initialize network}
			\item {Join network}
			\item {Leave network}
			\item {Broadcast message}
			\item {Register receive message listener}
		\end{itemize}
		In this section, we discuss how we plan to support these functionalities. But before starting with that, we should have a look at
		what a network is to us and how it is defined.\\
		Our system is supposed to be fully symmetric, i.e. there is no device (node) in the network with a special task, all nodes execute the same code. In particular, all nodes can send and receive message at any point in time new nodes can join the network by extending it at any node that is already integrated in the network.\\
		So, what is a network in our case? Let the pair $a(A,B)$ denote an established connection between node $A$ and $B$. We consider the known network of a node $C$ to be the set of all nodes $N_i, (1 \leq i \leq \text{\# of nodes})$ for which holds:  \\
		There is a chain of established connections for some $n$ and a given timeout $t$
		\begin{displaymath}
			a_0(N_{i},N_{j_0}) \circ a_1(N_{j_0},N_{j_1}) \circ \ldots \circ a_{n+1}(N_{j_n},C)
		\end{displaymath}
		such that $a_k$ happened before $a_{k+1}$ for all $k \leq n$ and $a_0$ is not older than the timeout $t$ allows. \\
		Informally, the network as seen by a given device $D$ consists of all nodes whose signal could reach the device within the predefined timeout. \\ \\
			
			To implement the functionalities described at the beginning, we establish connections over WiFi Direct and use a few data structures. Since we use these data structures later in our overview, we will explain it now. \\

% TODO: I think we should have a bulk of messages here, not a single message
			A message sent between two nodes is composite with a Header containing the LC- and the ACK-Table, as well as the content of the message, shown in the figure below. \\

		Message:
			\begin{center}
				\begin{tabular}{ | l |}
					\hline
					Last Contact Table \\ \hline
					ACK-Table \\ \hline
					Content \\ \hline
				\end{tabular}
			\end{center}


As a second data structure we have the Acknowledgement, seen below, which is just a message with no content. \\


			Acknowledgement:
			\begin{center}
				\begin{tabular}{ | l |}
					\hline
					Last Contact Table \\ \hline
					ACK-Table \\ \hline
				\end{tabular}
			\end{center}
			
In the two figures above we showed the form of the messages, which being sent. In their headers, they contain two tables, namely the Last Contact Table and the ACK-Table. We will start to show the Last Contact Table. \\

Each node in the network has a local Last Contact Table. This table has entries in form of ($N_{i}$, $T_{i}$), where N are nodes in the network and T are the corresponding timestamps. The timestamp represents the time when the node was last present in the network. That means that the timestamp is updataed each time the owner node hears from another node or it updates it's own LC-Table with one it got from another node. \\

			Last Contact Table:
			\begin{center}
				\begin{tabular}{ | l | l |}
					\hline
					$N_{1}$ & $T_{1}$ \\ \hline
					$N_{2}$ & $T_{2}$ \\ \hline
					\vdots & \vdots \\ \hline
					$N_{n}$ & $T_{n}$ \\ 
					\hline
				\end{tabular}
			\end{center}
			

The ACK-Table describes which Receiver nodes got a message from a particular Sender node. The table contains in the first column the Sender nodes and in the first row the Receiver nodes. Each entry in the table (except for the first row and the first column) contains a sequence number of the message from a Sender node, which the Receiver last received. \\
Important to mention is, since the API is based on a decentralized system, that the table only shows the view seen by the owner of the table at a given time. \\
			ACK-Table:
			\begin{center}
				\begin{tabular}{ l | l | l | l | l | l |}
					\multicolumn{5}{c}{Receiver}\\
					\cline{2-6}
					$\multirow{5}*{\rotatebox{90}{Sender}}$ & $\cellcolor[gray]{0.65}$ & $N_{1}$ & $N_{2}$ & $\hdots$ & $N_{n}$ \\ \cline{2-6}
					& $N_{1}$ & $\cellcolor[gray]{0.65}$ &  &  $\hdots$ &  \\ \cline{2-6}
					& $N_{2}$ &  & $\cellcolor[gray]{0.65}$ & $\hdots$ &  \\ \cline{2-6}
					& $\vdots$ &  &  & $\ddots$ &   \\ \cline{2-6}
					& $N_{n}$ & & & $\hdots$ & $\cellcolor[gray]{0.65}$ \\ \cline{2-6}
				\end{tabular}
			\end{center}
	
Now that we have discussed the involved data structures and how we define network, we can have a look at the actual implementation plan. \\
To \textbf{initialize} the network, we have to $\ldots$ (OPEN QUESTION: Distributed initializing and recursive merging? Or, single root and recursive cloning? Or initializing phase to connect all initially known devices?) \\
\textbf{Joining a network} for a single node is done by connecting with a node in the netwok and receving its LC- and ACK-table. Then we can extend both tables with a new node, for the entries in the cell we just copy the data from the node we connected to. \\ 
(OPEN QUESTION: Do we want to allow merging of networks? This is also dependant on the open question above.) \\
\textbf{Leaving a network} is not reall necessary for correctness, but it can make the whole network more efficient if nodes notify the network when they are not interested in the messages anymore. (OPEN QUESTION: Special signal that tells all nodes to delete that node? Or, setting the seq number to $\infty$ and only removing it after some time? Other ideas?)\\
\textbf{Broadcasting} a message (is also an OPEN QUESTION but here is my suggestion,) is done by adding the message to the local buffer and then invoking the send mechanism. The send mechanism goes through the local buffer and determines the messages which have not reached its neighbours. Neighbours are all those nodes which are currently visible. If there are neighbours which do not have all the locally buffered messages, then we send it all missing messages.\\
The \textbf{message listener} provided by the client will be called whenever a new message arrived. Right after calling the listener in a new thread, we can mark in the ACK-Table that we received that message.  \\


			
		
		\subsection{Emergency App}
		Claude, Alessandro \\
The main idea of this application is to provide emergency services even if a cellular connection cannot be directly established. \\
Users have to enter some personal data (name, address, birth date, insurance number (optional), allergies (optional) etc.) when launching the App. \\
Whenever a user gets into an unpleasant situation, he/she can set off an emergency message via the App (Graphic - Button Press). \\
The message contains the user's personal data, as well as his/her GPS coordinates at the time the emergency message was successfully sent. \\
Emergency App takes care of forwarding the message to a PoH (Point of Help). If cellular network is available, the emergency message is set off directly. \\
What if there is no direct connection? As soon as another user of the App is reachable via the WiFi Direct API, the emergency message is sent to that user who immediately gets notified that someone needs help. In case the new user is capable of a network connection, the message is sent to a PoH via his/her network connection. If not, the message is forwarded to another reachable user of the app. The message propagates across the growing chain of WiFi Direct connections and is flooded across the resulting network until a direct connection to a PoH can be established (SMS, TCP segment). The PoH then acks the message and the ACK is propagated along the network of users to stop the flooding and tell the victim that help is on the way. \\
Moreover, users on the WiFi direct chain get an estimation of the cardinal direction of where the emergency message was set off relative to their position in order to administer first aid.
However, if location services are not available to the victim (i.e. due to being stuck in a tunnel or cave), the first node on the emergency chain which can determine its GPS location puts it onto the message. This gives a reasonable approximation of the victim's location. \\
		
		\subsection{Chat App}
		Joel, Pascal
			The Chat App ensures end to end encrypted messages via peer-to-peer connection through the flooding API. Encrypting and Decrypting messages is done public key cryptography. The keys are generated by the user and shared by QR codes that have to be scanned from the receiver.\\
			If the receiver's network is not connected to the sender's network the messages are buffered and will be sent to the receiver later when the receiver's and the sender's network are connected. The receiver is able to get as many messages as are stored in the buffer.\\
			When first starting the App the user has to enter his name and generate his public and private key. After generating the key the user is able to scan public keys from other members or provide his own public key for scanning. Upon scanning a new chat is displayed in the chat-list and a reminder appears to scan the public key of the chat partner. \\
			Pressing on a chat in the chat-list opens a chat to write and read messages.\\

		% Describe system setup, components, external libraries, hardware etc.
	
	\section{Requirements}
		Several choices have to be made that limit the reach of our application, in order to keep the project simple enough for the given time frame. Perhaps the choice that limits us most is using the Wi-Fi Peer-to-Peer API in Android. It constrains us to devices that have at least Android 4 (API level 14) installed and that have hardware capable of Wi-Fi Direct communication.\cite{P2PAPIGuide}\\
For reading and generating QR codes we will use the ZXing project\cite{ZXing}. We will prompt user to install the ZXing Barcode Scanner app if not already available. It's installation size is in the neighbourhood of 1MB depending on the device, so the size is not a problem. The app requires API level 16, access to the Google Play Store and of course a device with a camera. However we provide an alternative, if somewhat arduous, method of copying the public keys by typing them in manually.\\
Beyond that we will use only standard Java and Android libraries so no further limitations apply to the system software.\\
We will develop three apps, one being a wrapper around the core networking service, with a simple interface to enter some configuration details, the others being the emergency app and the chat app. They are going to use an API to access network functionality from the first. Each app individually will be rather lightweight so storage concerns should be fairly insignificant since we don't include much audiovisual content in any of these components. In case the users wish to use one app with one group of people and the other with another, they will have to reconfigure their network association in the wrapper app repeatedly.\\
We are depending on users to judge the performance of the network on their respective devices and make sure they keep their networks small enough before scaling issues make it unusable.\\

	
		
		
	
	\section{Work Packages}
		% Breakdown the work to subtasks to meet the project requirements.
	% Define and describe these tasks.
	
		\subsection{API}
			Jakob, Manuel

\begin{enumerate}
% @Manu: Hey Manu, i schrib do gad ine was i ändere, wenns glese hesch und kei Iwänd hesch, bitte gad wieder lösche ;)

% @Manu: Das item hani no hinzuegfüegt, will mer da bi de Milestones denn sicher bruchet
\item Define all public function signatures of our API and hand it to the other group members
\item Establish Peer-to-Peer connection with WiFi Direct
\item Build data structures for LC-Table and ACK-Table
% @Manu: Zwei neue items
\item Implement network initialization with two nodes
\item Write code for nodes to join an existing network

\item Implement functions to update tables
\item Build message and parse message
% @Manu: I würd do no kein gnaui Funktionenäme verwende, die müemer zerscht no definiere, also würdi do kei `Code' ine tue
% OLD \item Implement broadcast\_message
\item Implement message sending (broadcast)

% @Manu: send_message - Bruchet mer da würklich? Inwiefern bringt üs da überhaup öpiis, wenn mer wüsset das e Nochricht nurzu eim mues? Verschicke tümers doch eh allne, damits so schnell wie möglich achunt. Und wie chömmer da speichere ohni dass mer üsen ACK-Table kaputt machet, bruchet mer en Zweite ACK-Table extra für da? I wär defür, das mer da weglönd, aber me chönds gern bespreche, je nochdem sötter mer au no di anderne froge was meinet.
% OLD\item Implement send\_message

% OLD: \item Implement receive\_message (with message listener)
\item Implement message receiving (with message listener)
\item Build data structure for local buffer
% @Manu: Au no neu:
\item Implement buffer entry replacement strategy
\item Remove old nodes from network according to a timeout specified by the client
\item Correct reconnection
\item Request all buffered messages from a node
\item Unsubscribing nodes from the network
\end{enumerate}


		
		\subsection{Emergency App}
			Claude, Alessandro
			\begin{enumerate}
				\item Main Activity with "request help" button. Button is only clickable if personal information is entered and location services are turned on. On button click the user can select what kind of emergency case it is.
				\item Settings Activity which stores personal information such as Name, insurance numbers, allergies, etc.
				\item Notification Activity which shows a relayed emergency request on the users phone including walk directions to find the requester.
				\item A webserver which distributes the request to the specific emergency services in charge.
			\end{enumerate}
		
		\subsection{Chat App}
			Joel, Pascal
			\begin{enumerate}
				\item Create a MainActivity with clickable list of chats ordered by activity. Each chat should display how many messages are unread.
				\item Add a overflow menu with "Preferences", "Show Key", "Add Chat", "Go Offline" buttons.
				\item Implement a service that handles message state, address book state, receiving messages including decryption, notification to be started which starts on app start if not running.
				\item Chat, address book and own keys have to be stored in separate files, when the service is shut down.
				\item ChatActivity: Chat window, with message list left and right aligned, depending on sender, ordered descending in age
				\item Add an activity to generate a public-private key pair with java.crypto.
				\item Using ZXing library make two activities, one for displaying keys and one for scanning them.
				\item Add and activity for initial key generation and name entry.
				\item In preference menu add two options to enable sound and vibration for notification. Furthermore add an option to generate a new key.
			\end{enumerate}
			
	% Breakdown the work to subtasks to meet the project requirements.
	% Define and describe these tasks.
	
	
	\section{Milestones}
		% The milestones section provides a work plan for carrying out the project.
% This is your schedule for getting the project done.
% Clearly state how the work packages will be distributed among the team members.
First of all the public function signatures of our API are defined and handed to the other group members that they can start with the Emergency App and the Chat App. Then the API team works at the remaining work packages and the other group members can start with their work on the emergency app and the chat app. The emergency app team will partially support the API team until the work packages 1 to 9 are met.\\[3mm]
Before the emergency app and the chat app can be tested the API has to be finished because the two apps rely on the message forwarding of the API. \\[3mm]
Schedule: \\
\begin{center}
\scalebox{.95}{
\begin{tabular}{|l|l|l|}
		\hline
		Date: & Subject to finish: & Responsible: \\ \hline
		20 Nov & function overview API & Manuel, Jakob \\
		24 Nov & Emergency App UI complete & Alessandro, Claude\\
		25 Nov & chat app up to WP3 complete & Joel, Pascal \\
		4 Dec & chat app up to WP5 complete & Joel, Pascal \\
		4 Dec & API: Basic send/recv. (up to WP9) & Manuel, Jakob \\
		10 Dec & Observable API behavior is stable & Manuel, Jakob \\
		11 Dec & chat app complete for testing & Joel, Pascal \\
		11 Dec & Emergency App: able to set off & \\ & and display requests & Alessandro, Claude\\
		14 Dec & Emergency App: Webservice for & \\ & distribution of requests running & Alessandro, Claude\\
		18 Dec & all tasks complete & all\\
		\hline
	\end{tabular}
}

\end{center}
		% The milestones section provides a work plan for carrying out the project.
		% This is your schedule for getting the project done.
		% Clearly state how the work packages will be distributed among the team members. 
		Pascal
	
	
	\bibliographystyle{abbrv}
	\bibliography{report}
	
\end{document}

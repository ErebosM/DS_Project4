% This is the core of the proposal.
	% It is where you spell out your technical plan and explain the project design.
	% Expected evaluation/demonstration issues would also be addressed in this section.
	% Use helpful figures such as~\rfig{example} and~\rfig{system-overview},
	% explain the figures in the text where you reference them. 
		\subsection{API}
		The API will offer the following functionalities to clients:
		\begin{itemize}
			\item {Initialize network}
			\item {Join network}
			\item {Leave network}
			\item {Broadcast message}
			\item {Register receive message listener}
		\end{itemize}
		In this section, we discuss how we plan to support these functionalities. But before starting with that, we should have a look at
		what a network is to us and how it is defined.\\
		Our system is supposed to be fully symmetric, i.e. there is no device (node) in the network with a special task, all nodes execute the same code. In particular, all nodes can send and receive message at any point in time new nodes can join the network by extending it at any node that is already integrated in the network.\\
		So, what is a network in our case? Let the pair $a(A,B)$ denote an established conection between node $A$ to node $B$. We consider the known network of a node $C$ to be the set of all nodes $N_i, (1 \leq i \leq \text{\# of nodes})$ for which holds:  \\
		There is a chain of established connections for some $n$ and a given timeout $t$
		\begin{displaymath}
			a_0(N_{i},N_{j_0}) \circ a_1(N_{j_0},N_{j_1}) \circ \ldots \circ a_{n+1}(N_{i_n},C)
		\end{displaymath}
		such that $a_k$ happened before $a_{k+1}$ for all $k \leq n$ and $a_0$ is not older than the timeout $t$ allows. \\
		Informally, the network as seen by a given device consits of all nodes whose signal could reach the device within the predefined timeout. \\
			
			Last Contact Table:
			\begin{center}
				\begin{tabular}{ | l | l |}
					\hline
					$N_{1}$ & $T_{1}$ \\ \hline
					$N_{2}$ & $T_{2}$ \\ \hline
					\vdots & \vdots \\ \hline
					$N_{n}$ & $T_{n}$ \\ 
					\hline
				\end{tabular}
			\end{center}
			
			ACK-Table:
			\begin{center}
				\begin{tabular}{ l | l | l | l | l | l |}
					\multicolumn{5}{c}{Receiver}\\
					\cline{2-6}
					$\multirow{5}*{\rotatebox{90}{Sender}}$ & $\cellcolor[gray]{0.65}$ & $N_{1}$ & $N_{2}$ & $\hdots$ & $N_{n}$ \\ \cline{2-6}
					& $N_{1}$ & $\cellcolor[gray]{0.65}$ &  &  $\hdots$ &  \\ \cline{2-6}
					& $N_{2}$ &  & $\cellcolor[gray]{0.65}$ & $\hdots$ &  \\ \cline{2-6}
					& $\vdots$ &  &  & $\ddots$ &   \\ \cline{2-6}
					& $N_{n}$ & & & $\hdots$ & $\cellcolor[gray]{0.65}$ \\ \cline{2-6}
				\end{tabular}
			\end{center}
			
			Message:
			\begin{center}
				\begin{tabular}{ | l |}
					\hline
					Last Contact Table \\ \hline
					ACK-Table \\ \hline
					Content \\ \hline
				\end{tabular}
			\end{center}
			
			Acknowledgement:
			\begin{center}
				\begin{tabular}{ | l |}
					\hline
					Last Contact Table \\ \hline
					ACK-Table \\ \hline
				\end{tabular}
			\end{center}
			
		
		\subsection{Emergency App}
		Claude, Alessandro \\
The main idea of this application is to provide emergency services even if a cellular connection cannot be directly established. \\
Users have to enter some personal data (name, address, birth date, insurance number (optional), allergies (optional) etc.) when launching the App. \\
Whenever a user gets into an unpleasant situation, he/she can set off an emergency message via the App (Graphic - Button Press). \\
The message contains the user's personal data, as well as his/her GPS coordinates at the time the emergency message was successfully sent. \\
Emergency App takes care of forwarding the message to a PoH (Point of Help). If cellular network is available, the emergency message is set off directly. \\
What if there is no direct connection? As soon as another user of the App is reachable via the WiFi Direct API, the emergency message is sent to that user who immediately gets notified that someone needs help. In case the new user is capable of a network connection, the message is sent to a PoH via his/her network connection. If not, the message is forwarded to another reachable user of the app. The message propagates across the growing chain of WiFi Direct connections and is flooded across the resulting network until a direct connection to a PoH can be established (SMS, TCP segment). The PoH then acks the message and the ACK is propagated along the network of users to stop the flooding and tell the victim that help is on the way. \\
Moreover, users on the WiFi direct chain get an estimation of the cardinal direction of where the emergency message was set off relative to their position in order to administer first aid.
However, if location services are not available to the victim (i.e. due to being stuck in a tunnel or cave), the first node on the emergency chain which can determine its GPS location puts it onto the message. This gives a reasonable approximation of the victim's location. \\
		
		\subsection{Chat App}
		Joel, Pascal
			The Chat App ensures end to end encrypted messages via peer-to-peer connection through the flooding API. Encrypting and Decrypting messages is done public key cryptography. The keys are generated by the user and shared by QR codes that have to be scanned from the receiver.\\
			If the receiver's network is not connected to the sender's network the messages are buffered and will be sent to the receiver later when the receiver's and the sender's network are connected. The receiver is able to get as many messages as are stored in the buffer.\\
			When first starting the App the user has to enter his name and generate his public and private key. After generating the key the user is able to scan public keys from other members or provide his own public key for scanning. Upon scanning a new chat is displayed in the chat-list and a reminder appears to scan the public key of the chat partner. \\
			Pressing on a chat in the chat-list opens a chat to write and read messages.\\

% This is the core of the proposal.
% It is where you spell out your technical plan and explain the project design.
% Expected evaluation/demonstration issues would also be addressed in this section.
% Use helpful figures such as~\rfig{example} and~\rfig{system-overview},
% explain the figures in the text where you reference them. 
\subsection{API}
	The API will offer the following functionalities to clients:
	\begin{itemize}
		\item {Initialize network}
		\item {Join network}
		\item {Broadcast message}
		\item {Register receive message listener}
	\end{itemize}
	
	In this section, we discuss how we plan to support these functionalities. But before starting with that, we should have a look at	what a network is to us and how it is defined.\\
	\subsubsection{Definition of a network}
	Our system is supposed to be fully symmetric, i.e. there is no device (node) in the network with a special task, all nodes execute the same code. In particular, all nodes can send and receive message at any point in time. New nodes can join the network by extending it at any node that is already integrated in the network. Furthermore, seperatly created networks should be able to be merged into one network.\\
	To define our network accurately, let the ordered pair $a(A,B)$ denote an established connection (using WiFi Direct) between node $A$ and $B$. Each such pair is associated with the timestamp when the connection establishment happened. There will usually be many such establishments with different timestamps that involve the same two nodes, because the nodes are constantly building up and droping connections to the nodes in reachable distance. We consider the known network of a node $C$ to be the set of all nodes $N_i, (1 \leq i \leq \text{\# of nodes})$ for which holds:  \\
	There is a chain of established connections for some $n$ and a given timeout $t$
	\begin{displaymath}
		a_0(N_{i},N_{j_0}) \circ a_1(N_{j_0},N_{j_1}) \circ \ldots \circ a_{n+1}(N_{j_n},C)
	\end{displaymath}
	such that $a_k$ happened before $a_{k+1}$ for all $k \leq n$ and $a_0$ is not older than the timeout $t$ allows. \\
	Informally, the network as seen by a given device $D$ consists of all nodes whose signal could reach $D$ within the predefined timeout. \\ 
	
	\subsubsection{Data structures}
	To implement the functionalities described at the beginning, we use a few data structures. We will explain them now. \\

	A message sent between two nodes is composite with a Header containing the LC- and the ACK-Table, as well as the content of a number of messages, shown in the figure below. \\

	With Messages:
		\begin{center}
			\begin{tabular}{ | l |}
				\hline
				Last Contact Table \\ \hline
				ACK-Table \\ \hline
				Content of Message 1\\ \hline
				\vdots\\ \hline
				Content of Message n\\ \hline
			\end{tabular}
		\end{center}


As a second data structure we have the Acknowledgement, seen below, which is just a message with no content. \\


	Acknowledgement:
		\begin{center}
			\begin{tabular}{ | l |}
				\hline
				Last Contact Table \\ \hline
				ACK-Table \\ \hline
			\end{tabular}
		\end{center}
			
In the two figures above we showed the form of the messages, which are being sent. In their headers, they contain two tables, namely the Last Contact Table and the ACK-Table. We start by explaining the Last Contact Table. \\

Each node in the network has a local Last Contact Table. This table has entries in form of ($N_{i}$, $T_{i}$), where N are nodes in the network and T are the corresponding timestamps. The timestamp represents the time when the node was last present in the network. That means that the timestamp is updated each time the owner node hears from another node and updates its LC-Table with the earlier timestamps. \\

	Last Contact Table:
		\begin{center}
			\begin{tabular}{ | l | l |}
				\hline
				$N_{1}$ & $T_{1}$ \\ \hline
				$N_{2}$ & $T_{2}$ \\ \hline
				\vdots & \vdots \\ \hline
				$N_{n}$ & $T_{n}$ \\ 
				\hline
			\end{tabular}
		\end{center}

When we filter out nodes from the LC-table which are older than a specified timeout $t$, then the list of nodes in such a local table corresponds to our definition of a network as seen from a specific node. To build up the next data structure, we will also need this list of nodes considered to be in the known network. \\

The ACK-Table describes which Receiver nodes got a message from a particular Sender node. The table contains in the first column the Sender nodes and in the first row the Receiver nodes. Each entry in the table (except for the first row and the first column) contains a sequence number of the message from a Sender node, which the Receiver last received. \\
Important to mention is, since the API is based on a decentralized system, that the table only shows the view seen by the owner of the table at a given time. \\
	ACK-Table:
		\begin{center}
			\begin{tabular}{ l | l | l | l | l | l |}
				\multicolumn{5}{c}{Receiver}\\
				\cline{2-6}
				$\multirow{5}*{\rotatebox{90}{Sender}}$ & $\cellcolor[gray]{0.65}$ & $N_{1}$ & $N_{2}$ & $\hdots$ & $N_{n}$ \\ \cline{2-6}
				& $N_{1}$ & $\cellcolor[gray]{0.65}$ &  &  $\hdots$ &  \\ \cline{2-6}
				& $N_{2}$ &  & $\cellcolor[gray]{0.65}$ & $\hdots$ &  \\ \cline{2-6}
				& $\vdots$ &  &  & $\ddots$ &   \\ \cline{2-6}
				& $N_{n}$ & & & $\hdots$ & $\cellcolor[gray]{0.65}$ \\ \cline{2-6}
			\end{tabular}
		\end{center}
	We will use this ACK-table to keep track of which messages have reached all nodes, thus can be removed from the local buffer, while other messages should be kept so we can deliver it to the nodes which have not received it, yet.	
	
	\subsubsection{Implementation}
Now that we have discussed the involved data structures and how we define network, we can have a look at the actual implementation plan. Devices (nodes) establish connections over WiFi Direct and the protocol we use to communicate will be TCP.\\
To \textbf{initialize} the network, we have to create a ACK- and a LC-table with only our own entry. Then we simply contact other nodes and perform a merge of the tables whenever we encounter tables with more entries. \\
\textbf{Joining a network} for a single node is basically just initializing the network as if it was new. However, we might want to ask around for old messages that the other nodes have still stored in their local buffers. \\ 
\textbf{Broadcasting} a message is done by adding the message to the local buffer and then invoking the send mechanism. The send mechanism goes through the local buffer and determines the messages which have not reached its neighbours. Neighbours are all those nodes which are currently visible. If there are neighbours which do not have all the locally buffered messages, then we send it all missing messages.\\
The \textbf{message listener} provided by the client will be called whenever a new message arrived. Right after calling the listener in a new thread, we can mark in the ACK-Table that we received that message.  \\
To allow above functionalities, we have to implement a way of merging networks, i.e. merging the LC- and ACK-tables of different networks. Further, we have to build up WiFi Direct connections with all nodes in our reach. We can build up the LC-table in a way that we know the MAC address of all nodes, therefore we will know which visible devices are part of our network. With those we will build up a connection whenever we receive data that, according to our ACK-table, has not reached this node. If the entry of a node in our LC-table is getting older than a predefined threshold, then we also try to establish the connection again, to keep the information about the network up to date. \\
Devices which are not part of our network have to be considered as well, they could be in a network that could be merged with ours. Therefore we keep a list of devices that are not part of our network, too. For each such device, we store the timestamp when we attempted to merge the last time. When this entry is getting old and the device is still visible, we retry. \\

\subsubsection{Target client applications}
	Even though our API can serve as backend to a variety of different application, some of our design choices are based on assumptions on how it will be used. \\
	First, we assume the network to consists of a relativly small number of nodes, typically not more than 20 nodes. We will support more nodes, but the performance might be very poor and we will \textit{not} test our code for good scalability to hundreds of nodes. \\
	\textbf{TODO}: Second, (What ever we decide about the many networks on a single device problem)
		
\subsection{Emergency App}
The main idea of this application is to provide emergency services even if a cellular connection cannot be directly established. \\
Users have to enter some personal data (name, address, birth date, insurance number (optional), allergies (optional) etc.) when launching the App. \\
Whenever a user gets into an unpleasant situation, he/she can set off an emergency message via the App (Graphic - Button Press). \\
The message contains the user's personal data, as well as his/her GPS coordinates at the time the emergency message was successfully sent. \\
Emergency App takes care of forwarding the message to a PoH (Point of Help). If cellular network is available, the emergency message is set off directly. \\
What if there is no direct connection? As soon as another user of the App is reachable via the WiFi Direct API, the emergency message is sent to that user who immediately gets notified that someone needs help. In case the new user is capable of a network connection, the message is sent to a PoH via his/her network connection. If not, the message is forwarded to another reachable user of the app. The message propagates across the growing chain of WiFi Direct connections and is flooded across the resulting network until a direct connection to a PoH can be established (SMS, TCP segment). The PoH then acks the message and the ACK is propagated along the network of users to stop the flooding and tell the victim that help is on the way. \\
Moreover, users on the WiFi direct chain get an estimation of the cardinal direction of where the emergency message was set off relative to their position in order to administer first aid.
However, if location services are not available to the victim (i.e. due to being stuck in a tunnel or cave), the first node on the emergency chain which can determine its GPS location puts it onto the message. This gives a reasonable approximation of the victim's location. \\
		
\subsection{Chat App}
The Chat App ensures end to end encrypted messages via peer-to-peer connection through the flooding API. Encrypting and Decrypting messages is done public key cryptography. The keys are generated by the user and shared by QR codes that have to be scanned from an other user which acts as a chat partner.\\
If the receiver's network is not connected to the sender's network the messages are buffered and will be sent to the receiver later when the receiver's and the sender's network are connected. The receiver is able to get as many messages as are stored in the buffer of the API.\\
When first starting the App the user has to enter his name and generate his public and private key. After generating the key the user is able to scan public keys from other members or provide his own public key for scanning. After scanning a public key the App reminds to scan the public key of the other chat partner. If only one chat partner has scanned the public key it will only be possible to send messages in one direction (one is acting always as a sender, the chat partner is acting always as a receiver).\\
Pressing on a chat in the chat-list opens a chat to write and read messages. The messages are simple text messages with a maximum length which is determined by the API. \\